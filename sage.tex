\documentclass[12pt,a4]{article}
\usepackage{chemarr}
\usepackage{color}
\usepackage{url}
\usepackage[pdftex]{hyperref}

\usepackage[pdftex]{graphicx}

\begin{document}

\graphicspath{{figs/}}

\DeclareGraphicsExtensions{.pdf, .jpg, .png}

\title{\centerline{{\underline{\texttt{SAGE} Documentation}}}}
\author{Matt Robinson}
\date{15th March 2011}
\maketitle

\section{Installation}

See README.

\section{Introduction}

This document is a work in progress, but I thought something was better than nothing.

SAGE (See A Geant Event) is an event viewer for Geant4
It reads \texttt{wrl} (VRML) files containing tracks and detector components and displays them.
The display is based on OpenGL and performs better on decent graphics cards.  Performance suffers on slow graphics cards and over X-Window forwarding, but it should still work well enough.

\begin{figure}[htb]
\includegraphics[width=0.95\textwidth]{gui.pdf}
\end{figure}

\section{Loading VRML (wrl) Files}

To open in the current window, choose 'Open' from the 'File' menu.  To open in a new window, choose 'New' from that menu.
Double-click to open should work on Mac or Windows, or after proper set-up on Linux.

\section{Navigation}

Navigation is achieved by holding down the left mouse button over the display and dragging.

There are 3 primary navigation modes: 'Move', 'Pan', and 'Zoom/Twist'.
These are accessible from the buttons at the bottom of the window, or from the 'Navigation' menu.

Move: shift the detector left/right or up/down;
Pan: rotate the detector about the vertical or horizontal axis;
Zoom/Twist: shift the detector forward/back or rotate about the depth axis.

'Origin' is a special mode which allows the user to change the centre of rotation.  It is useful to move this to the centre of the part of the event/geometry which is of interest.  Then this will stay in the centre of the view when using 'Pan' navigation mode.

Point the mouse at something and it will be identified by the display of some text in the bottom right of the display.

The keyboard arrow keys can be used to navigate if you prefer.

Alternatively, there are keyboard shortcuts which work without switching modes

\vspace{0.5cm}
\begin{tabular}{|c|c|}
\hline
Key & Action \\
\hline
A & Move left \\
D & Move right \\
W & Move up \\
S & Move down \\
\hline
E & Pan left \\
T & Pan right \\
R & Pan up \\
F & Pan down \\
\hline
G & Twist left \\
J & Twist right \\
Y & Zoom in \\
H & Zoom out \\
\hline
U & Shrink origin sphere \\
O & Enlarge origin sphere \\
I & Move origin out \\
K & Move origin in \\
\hline
\end{tabular}
\vspace{0.5cm}

If you find that the keyboard shortcuts are not working, it's probably because the `focus' is not on the 3d display.  Hit `Tab' to shift the focus, or click on the 3d display.

\section{Recording what you see}

The 'Record' menu allow you to record either a snapshot of the display, or a short video clip (up to 25 seconds).  Video clips are limited by RAM availability.

After recording a video clip, you will have a whole bunch of png files for each frame.  A script is generated for converting this to a Windows friendly avi or a modern mp4.  mencoder is needed for the conversion to avi or to mp4.  mp4 conversion also required MP4Box from gpac.

\section{Viewing Modes}

There are several viewing modes on the 'View' menu.
There is a full screen mode, Cmd/Ctrl+F to get in or out of this mode.

You can choose to show the detector as a wire-frame, in full, or both or neither.  The default is to show the detector wire-frame.

\section{Detector Geometry}

From the ``Detector'' list, you can turn bits of the detector on and off.  This may be useful if you want to get a clear view of components which are to some degree obscured.

\section{Configuring Geant4}

Geant4 should be built with vrml support.  This is normally how it's built, be some may find it necessary to recompile.

Set the environment variable \texttt{G4VRMLFILE\_VIEWER} to \texttt{sage} and set \texttt{G4VIS\_USE\_VRML} to \texttt{1}.

In your geant4 macro file include the following: \\
\texttt{/vis/scene/create} \\
\texttt{/vis/open VRML2FILE} \\
\texttt{/vis/scene/add/trajectories} \\
\texttt{/vis/scene/endOfEventAction accumulate} \\
and \\
\texttt{/vis/viewer/flush} \\
or \\
\texttt{/run/beamOn 1} \\

\vspace{2cm}

\texttt{m.robinson@sheffield.ac.uk}

\end{document}
